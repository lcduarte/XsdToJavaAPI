\chapter{Problem Statement}
\label{cha:problem}

\sloppy

In the first chapter we presented \textit{template engines} and discussed their theoretical handicaps, in this chapter we will further analyze other limitations that are presented while using them in a practical setting. This analysis aims to show how fragile the usage of this type of solution can be and the problems that are inherited by using it.

\section{Motivation}
\label{sec:motivation}

Text has evolved with the advance of technology resulting in the creation of \textit{markup languages} \cite{markuplanguages}. Markup languages work by adding annotations to text, the annotations being also known as tags, that allow to add additional information to the text. Each markup language has its own tags and each of those tags add a different meaning to the text encapsulated within them. In order to use markup languages the users can write the text and add all the tags manually, either by fully writing them or by using some kind of text helpers such as text editors with IntelliSense\footnote{\url{https://www.techopedia.com/definition/24580/intellisense}} which can help diminish the errors caused by manually writing the tags. But even with text helpers the resulting document can violate the restrictions of the respective markup language because the editors do not actually enforce the language rules since there is not a process similar to a compile process that can either pass of fail. The most that a text editor can do is highlight the errors to the user.

\noindent
The most well known markup language is \ac{HTML}, which is highly used in Web applications. Other uses of the \ac{HTML} language are in emails, writing reports, etc.

\noindent
In the following we will present different problems resultng from the use of \textit{template engines} to build \ac{HTML} views. The examples provided in this section use eight different \textit{template engines}: Freemarker\cite{freemarker}, Handlebars\cite{handlebars}, Mustache\cite{mustache}, Pebble\cite{pebble}, Thymeleaf\cite{thymeleaf}, Trimou\cite{trimou}, Velocity\cite{velocity} and Rocker\cite{rocker}. These templates were used in experiment tests for different benchmarks presented in Chapter \ref{cha:deployment}.

\noindent
We will start by the most basic aspect that we expect from a \ac{XML} document, it should be well formed. Let us start with a very simple example as shown in Listing \ref{lst:wellformedex}.

\lstdefinestyle{problemex}{
  moredelim=**[is][\color{blue}]{@}{@},
  moredelim=**[is][\color{red}]{|}{|},
}

\lstset{language = java}

\bigskip

\begin{minipage}{\linewidth}
\begin{lstlisting}[caption={Badly Formed HTML Document}, label={lst:wellformedex}, style=problemex]
@<html@|>|
	@<!-- -->@
|</html>|
\end{lstlisting}
\end{minipage} 

\noindent
Let us imagine that for some typing mistake the red characters are missing, which means that the opening \texttt{<html>} tag is not properly written and that the \texttt{html} element does not have a matching closing tag. It would be expected that in the very least \textit{template engine} would issue an error while reading the file at run time. But every one of the \textit{template engines} used with this example have not issued any kind of error. This is problematic, because the error was not caught neither at compile time nor at run time. These kind of errors would only be observable either on a browser or by using any kind of external tool to verify the resulting \ac{HTML} page. This is the case where an internal \ac{DSL} such as the one presented in Listing \ref{lst:staticview} suppresses this problem since the responsibility of creating tags and properly open and closing them should be performed by the \ac{DSL} library and not by the person who is writing the template.

\noindent
The second problem that we are going to pinpoint is the use of \textit{context objects}. Every \textit{template engine} uses them, since it contains the information that the \textit{template engine} will use to fill out the \textit{placeholders} defined in the textual template file. But what problems arise from their usage?

\bigskip

\lstset{language=html, morekeywords={TableElement, String}}

\begin{minipage}{\linewidth}
\begin{lstlisting}[caption={HTML Template with Placeholders}, label={lst:contextobjs}, style=problemex]
<html>
    <body>
        <ul>
        {{#student}}
            <li>
                {{name}}
            </li>
            <li>
                {{number}}
            </li>
        {{/student}}
        </ul>
    </body>
</html>
\end{lstlisting}
\end{minipage} 

\noindent
The template of Listing \ref{lst:contextobjs} receives a \texttt{Student} object that contains a \texttt{name} and \texttt{number} fields. Most \texttt{template engines} use a \texttt{Map<String, Object>} as the \textit{context object}. In this case, a valid \texttt{context object} should look like the \texttt{Map} object created in Listing \ref{lst:contextobj}.

\bigskip

\lstset{language=java, morekeywords={Map, String, Object, HashMap, put, Student}}

\begin{minipage}{\linewidth}
\begin{lstlisting}[caption={Template Engine with a valid Context Object}, label={lst:contextobj}]
Map<String, Object> context = new HashMap<>();
context.put("student", new Student("Luis", 39378));
\end{lstlisting}
\end{minipage} 

\noindent
The previous example, Listing \ref{lst:contextobj}, is correct since there is one object in the \textit{context object} with the \texttt{student} key with an instance of a \texttt{Student} object, which contains a \texttt{name} and \texttt{number} fields, which corresponds with the usage performed in the template defined in Listing \ref{lst:contextobjs}. Yet in Listing \ref{lst:contextobj1} and Listing \ref{lst:contextobj2} we show another situation of illegal \texttt{context objects} which turn their use invalid by the template of Listing \ref{lst:contextobjs}.

\bigskip

\lstset{language=java, morekeywords={Map, String, Object, HashMap, put, Student}}

\begin{minipage}{\linewidth}
\begin{lstlisting}[caption={Template Engine with a Context Object with a wrong key}, label={lst:contextobj1}]
Map<String, Object> context = new HashMap<>();
context.put("teacher", new Student("Luis", 39378));
\end{lstlisting}
\end{minipage} 

\bigskip

\lstset{language=java, morekeywords={Map, String, Object, HashMap, put, Teacher}}

\begin{minipage}{\linewidth}
\begin{lstlisting}[caption={Template Engine with a Context Object with a wrong type}, label={lst:contextobj2}]
Map<String, Object> context = new HashMap<>();
context.put("student", new Teacher("MEIC", "ADDETC"));
\end{lstlisting}
\end{minipage} 

\noindent
The first \texttt{context object} of Listing \ref{lst:contextobj1} has a wrong key, \texttt{teacher}, whereas the template is expecting an object with the \texttt{student} key. The second \texttt{context object} of Listing \ref{lst:contextobj2} has the right key but has a different type, which does not match the fields expected by template of Listing \ref{lst:contextobjs}.

\noindent
With this information in mind how will the eight template engines react when receiving these two wrongly defined \textit{context objects}? The Rocker template engine is the only one which deals with it in a safe way since its template defines the type that will be received. Moreover its template file is converted in a Java class at compile time and its usages are all safe regarding the \textit{context object}, because the Java compiler validates if the object received as \textit{context object} matches the expected type. The remaining seven \texttt{template engines} have no static validations. None of them issue any compile time warning. 

\noindent
Regarding runtime safety only Freemarker issues an exception with a similar example to Listing \ref{lst:contextobj1} and in the second case, Listing \ref{lst:contextobj2}, only Freemarker and Thymeleaf throw an exception. The remaining solutions ignore the fact that something that is expected is not there and delay the error finding process until the generated file is manually validated. 

\noindent
In this case the use of an internal \ac{DSL} suppresses this problem. Since the template would be defined as a Java function its \texttt{context object} would be represented by the method arguments, which have their type validated at compile time.

\noindent
Another improvement of using an internal \ac{DSL} over the use of \texttt{template engines} is the removal of language heterogeneity. For example, even for the simplest templates we have to atleast use three distinct syntaxes. In the following example we will use the Pebble \textit{template engine}, one who requires less verbose. In this example we define a template to write an \ac{HTML} document that presents the \texttt{name} of all the \texttt{Student} objects present in a \texttt{Collection} of \texttt{Student} as shown in Listing \ref{lst:listofnamestemplate}.

\bigskip

\lstset{language=html, morekeywords={TableElement, String}}

\begin{minipage}{\linewidth}
\begin{lstlisting}[caption={List of Student Names Template using Pebble}, label={lst:listofnamestemplate}]
<html>
	<body>
	    <ul>
			
		  	<li>{{student.name}}</li>
			
		</ul>  
  	</body>
</html>
\end{lstlisting}
\end{minipage} 

\noindent
In this template alone we need to use two distinct syntaxes, the \ac{HTML} language and the Pebble syntax to express that the template will receive a \texttt{Collection} that should be iterated  and create \texttt{li} tags containing the \texttt{name} field of the received type. Apart from the template definition we also need the Java code to generate the complete document, as shown in Listing \ref{lst:listofnamesjava}.

\bigskip

\lstset{language=java, morekeywords={PebbleEngine, Builder, build, getTemplate, StringWriter, Map, String, Object, HashMap, getListOfStudents, evaluate, toString, put}}

\begin{minipage}{\linewidth}
\begin{lstlisting}[caption={List of Student Names building in Java using Pebble}, label={lst:listofnamesjava}]
PebbleEngine engine = new PebbleEngine.Builder().build();
template = engine.getTemplate("templateName.html");
StringWriter writer = new StringWriter();

Map<String, Object> context = new HashMap<>();
context.put("students", getListOfStudents());

template.evaluate(writer, context);

String document = writer.toString();
\end{lstlisting}
\end{minipage} 

\noindent
Even though the template in Listing \ref{lst:listofnamestemplate} is simple the usage of multiple syntaxes introduces more complexity to the problem. If we escalate the complexity of the template and the number of different types used in the context object mistakes are bound to happen, which would be fine if the template engines gave any kind of feedback on errors, but we already shown that most errors are not reported. Let us take a peek of how this same template would be presented in the latest HtmlFlow version with the definition of the template in Listing \ref{lst:htmlflowlistofnamestemplate} and the template building in Listing \ref{lst:htmlflowlistofnamestemplatebuilding}.

\bigskip

\lstset{language=java, morekeywords={String, view, DynamicHtml, CurrentClass, render, getStudentList, Iterable, Student, html, body, ul, dynamic, forEach, text, li, getName}}

\begin{minipage}{\linewidth}
\begin{lstlisting}[caption={List of Student Names Template using HtmlFlow/xmlet}, label={lst:htmlflowlistofnamestemplate}, literate={º}{\textdegree}1]
static void studentListTemplate(DynamicHtml<Iterable<Student>> view,                                
                                Iterable<Student> students){
    view.html()
            .body()
                .ul()
                    .dynamic(ul -> 
                      students.forEach(student ->                   
                        ul.li().text(student.getName()).º()))
                .º()
            .º()
        .º();
}
\end{lstlisting}
\end{minipage} 

\bigskip

\lstset{language=java, morekeywords={String, view, DynamicHtml, CurrentClass, render, getStudentList, Iterable, Student, html, body, ul, dynamic, forEach, text, li, getName}}

\begin{minipage}{\linewidth}
\begin{lstlisting}[caption={List of Student Names building using HtmlFlow/xmlet}, label={lst:htmlflowlistofnamestemplatebuilding}, literate={º}{\textdegree}1]
String document = DynamicHtml.view(CurrentClass::studentListTemplate)
                             .render(getStudentsList());
\end{lstlisting}
\end{minipage} 

\noindent
With this solution we have a very compact template definition, where the context object, i.e. the \texttt{Iterable<Student> students}, is validated by the Java compiler in compile time which guarantees that any document generated by this solution will be valid since the program would not compile otherwise. This solution internally guarantees that the \ac{HTML} tags are created properly, having matching opening and ending tags, meaning that every document generated by this solution will be well formed regardless of the defined template. 

\noindent
Another quality of life improvement that we obtain by an internal \ac{DSL} is navigability. One aspect that is very common in template engines solutions is to define \textit{partial views} that can be reused in different views, but with regular text editors sometimes is hard to navigate back and forth between partial/regular templates. By using templates within the language we are able to quickly move between templates, since the template is either a method or a field and most \ac{IDE}s allow to quickly access both of them.

\section{Problem Statement}
\label{sec:problemstatement}

The problem that is being presented revolves around the handicaps of \textit{template engines}, the lack of compilation of the language used within the template, the performance \textit{overhead} that it introduces and the issues that it was when the complexity increases, as it was presented in the previous Section \ref{sec:templateengineshandicaps}. To tackle those handicaps we suggested the automated generation of a strongly typed \textit{fluent interface}. To show how that \textit{fluent interface} will effectively work we will now present a small example which consists on the \texttt{html} element, Listing \ref{lst:codegenerationexample}, described in \ac{XSD} of the \ac{HTML}5 language definition. The presented example is simplified for explanation purposes.

\bigskip

\lstset{
	language=XML,
	morekeywords={xs:element, name, xs:complexType, xs:choice, ref, xs:attributeGroup, xs:attribute, type}
}

\begin{minipage}{\linewidth}
\begin{lstlisting}[caption={<hmtl> Element Description in XSD},captionpos=b, ,label={lst:codegenerationexample}]
<xs:attributeGroup name="commonAttributeGroup">
    <xs:attribute name="someAttribute" type="xs:string">
</xs:attributeGroup>

<xs:element name="html">
    <xs:complexType>
        <xs:choice>
            <xs:element ref="body"/>
            <xs:element ref="head"/>
        </xs:choice>
        <xs:attributeGroup ref="commonAttributeGroup" />
        <xs:attribute name="manifest" type="xs:string" />
    </xs:complexType>
</xs:element>
\end{lstlisting}
\end{minipage}

\noindent
With this example there is a multitude of classes that need to be created, apart from the always present supporting infrastructure that will be presented in Section \ref{sec:supportinginfrastructure}. 

\begin{itemize}
	\item \texttt{Html} Element - A class that represents the \texttt{Html} element (Listing \ref{lst:htmlclass}), deriving from \texttt{AbstractElement}.
	\item \texttt{Body} and \texttt{Head} Methods - Both methods present in the \texttt{Html} class (Listing \ref{lst:htmlclass}) that add \texttt{Body} (Listing \ref{lst:bodyclass}) and \texttt{Head} (Listing \ref{lst:headclass}) instances to \texttt{Html} children list.
	\item \texttt{Manifest} Method - A method present in \texttt{Html} class (Listing \ref{lst:htmlclass}) that adds an instance of the \texttt{Manifest} attribute (Listing \ref{lst:manifestattributeclass}) to the \texttt{Html} attribute list.
\end{itemize}

\lstset{language=Java, morekeywords={AbstractElement, CommonAttributeGroup, Body, addChild, Head, String, AttrManifest, addAttr}}

\begin{minipage}{\linewidth}
\begin{lstlisting}[caption={Generated Html Element Class},captionpos=b,label={lst:htmlclass}]
class Html extends AbstractElement implements CommonAttributeGroup {
    public Html() { }
    
    public void accept(Visitor visitor){
		visitor.visit(this);    
    }
    
    public Html attrManifest(String attrManifest) {
        return this.addAttr(new AttrManifest(attrManifest));
    }
    
    public Body body() { return this.addChild(new Body()); }
        
    public Head head() { return this.addChild(new Head()); }
}
\end{lstlisting}
\end{minipage}

\begin{itemize}
	\item \texttt{Body} and \texttt{Head} classes - Classes for both \texttt{Body} (Listing \ref{lst:bodyclass}) and \texttt{Head} (Listing \ref{lst:headclass}) elements, similar to the generated \texttt{Html} class (Listing \ref{lst:htmlclass}). The class contents will be dependent on the contents present in the concrete \texttt{xsd:element} nodes.
\end{itemize}

\bigskip

\begin{minipage}{\linewidth}
\begin{lstlisting}[caption={Generated Body Element Class},captionpos=b,label={lst:bodyclass}]
public class Body extends AbstractElement {
    //Similar to Html, based on the contents of the respective
    //xsd:element node.
}
\end{lstlisting}
\end{minipage}

\bigskip

\begin{minipage}{\linewidth}
\begin{lstlisting}[caption={Generated Head Element Class},captionpos=b,label={lst:headclass}]
public class Head extends AbstractElement {
    //Similar to Html, based on the contents of the respective 
    //xsd:element node.
}
\end{lstlisting}
\end{minipage}

\begin{itemize}
	\item \texttt{Manifest} Attribute - A class that represents the \texttt{Manifest} attribute  (Listing \ref{lst:manifestattributeclass}), deriving from \texttt{BaseAttribute}.
\end{itemize}

\bigskip

\lstset{language=Java, morekeywords={String, BaseAttribute}}

\begin{minipage}{\linewidth}
\begin{lstlisting}[caption={Generated Manifest Attribute Class},captionpos=b,label={lst:manifestattributeclass}]
public class AttrManifestString extends BaseAttribute<String> {
    public AttrManifestString(String attrValue) {
        super(attrValue);
    }
}
\end{lstlisting}
\end{minipage}

\begin{itemize}
	\item \texttt{CommonAttributeGroup} Interface - An interface with default methods that add the group attributes to the concrete element (Listing \ref{lst:commonattributegroup}).
\end{itemize}

\bigskip

\lstset{language=Java, morekeywords={addAttr, Html, Element, String, SomeAttribute}}

\begin{minipage}{\linewidth}
\begin{lstlisting}[caption={Generated CommonAttributeGroup Interface},captionpos=b,label={lst:commonattributegroup}]
public interface CommonAttributeGroup extends Element {
    default Html attrSomeAttribute(String attributeValue) {
        this.addAttr(new SomeAttribute(attributeValue));
        return this;
    }
}
\end{lstlisting}
\end{minipage}

\noindent
By analyzing this little example we can observe how the \texttt{xmlet} solution implements one of its most important features that was lacking in the \textit{template engine} solutions, the user is only allowed to generate a tree of elements that follows the rules specified in the \ac{XSD} file of the given language, e.g. the user can only add \texttt{Head} and \texttt{Body} elements as children to the \texttt{Html} element and the same goes for attributes as well, the user can only add a \texttt{Manifest} or \texttt{SomeAttribute} objects as attribute. This solution effectively uses the Java compiler to enforce the specific language restrictions, most of them at compile time. The other handicaps are also solved, the template can now be defined within the Java language eradicating the necessity of textual files that still need to be loaded into memory and resolved by the \textit{template engine}. The complexity and flexibility issues are also tackled by moving all the parts of the problem to the Java language, it removes the necessity of additional syntax and now the Java syntax can be used to create the templates.

\newpage

\section{Approach}
\label{sec:approach}

The approach to achieve a solution was to divide the problem into three distinct aspects, as previously stated in Section  \ref{sec:thesisstatement}. 

\noindent
The XsdParser project will be an utility project which is needed in order to parse all the external \ac{DSL} rules present in the \ac{XSD} document into structured Java classes. 

\noindent
The XsdAsm is the most important aspect of the \texttt{xmlet} solution, since it is the aspect which will deal with the generation of all the \textit{bytecodes} that make up the classes of the Java \textit{fluent interface}. This project should translate as many rules of the parsed language definition, its \ac{XSD} file, into the Java language in order to make the resulting \textit{fluent interface} as much similar as possible to the language definition.

\noindent
The HtmlApi will be a representation of client aspect of \texttt{xmlet} solution. It is a concrete client of the XsdAsm project, it will use the \ac{HTML}5 language definition file in order to request of XsdAsm a strongly typed \textit{fluent interface}, named HtmlApi. This use case is meant to be used by the HtmlFlow library which will use HtmlApi to manipulate the \ac{HTML} language to write well formed documents.