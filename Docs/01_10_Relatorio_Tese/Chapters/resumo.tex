\abstractPT  % Do NOT modify this line

%Introdução - Informar, em poucas palavras, o contexto em que o trabalho se insere, sintetizando a problemática estudada. 
Actualmente a utilização de linguagens de \textit{markup} é recorrente no mundo da tecnologia, sendo o \ac{HTML} a linguagem mais utilizada graças à sua utilização no mundo da Web. Tendo isso em conta é necessário que existam ferramentas capazes de escrever documentos bem formados de forma eficaz. Actualmente essa tarefa é realizada por \textit{template engines}, tendo como base ficheiros externos com \textit{templates} de resposta, o que não garante que estes sejam bem formados e acrescenta o \textit{overhead} do carregamento do ficheiro para memória.

% Objectivo - Deve ser explicitado claramente. 
\noindent
O nosso objectivo é criar as ferramentas necessárias para gerar \textit{interfaces fluentes} tendo em conta a sua definição sintática, expressa em \ac{XSD}, garantindo que as restrições dessa mesma linguagem são verificadas. A geração de \textit{interfaces fluentes} deve ser automatizada de modo a evitar erro humano e tornar a geração de código mais rápida. Automatizando a geração das \textit{interfaces fluentes} cria-se também uma abordagem uniforme às diferentes linguagens de domínio utilizadas.

%Métodos - Destacar os procedimentos metodológicos adoptados. 
\noindent
Para alcançar os nossos objectivos vamos utilizar a linguagem Java para extrair informação sintática da linguagem do seu ficheiro de definição. Tendo essa informação em conta vão ser gerados \texttt{bytecodes} para refletir a definição da linguagem para a linguagem Java. Para implementar as restrições em Java é sempre prioritizada a validação de restrições em tempo de compilação, apenas validando em tempo de execução a informação que não existe aquando da geração da \textit{interface fluente}.

%Resultados - Destacar os mais relevantes para os objectivos pretendidos. 
\noindent
Comparando a solução desenvolvida com soluções semelhantes, incluindo \textit{template engines} e algumas soluções com inovações face à abordagem dos \textit{template engines}, obtemos resultados favoráveis. Verificamos que a solução sugerida é a mais eficiente em todos os testes feitos. Estes resultados são importantes, especialmente considerando que apesar de ser a solução mais eficiente introduz também a verificação das restrições da linguagem utilizada tendo em conta a sua definição sintática.

% Keywords of abstract in Portuguese
\begin{keywords}
XML, XSD, Geração Automática de Código, \texttt{interface} fluente.
\end{keywords}
% to add an extra black line
