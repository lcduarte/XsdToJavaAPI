\chapter{Existent Tools}
\label{cha:tools}

In this chapter we are going to introduce the \ac{XSD} language in order to provide a better understanding of the next chapters and also introduce some tools that were discovered during the development of this thesis. 

\section{XSD Language} % (fold)
\label{sec:xsd}

The \ac{XSD} language is a description of a type of \ac{XML} document. Its purpose is to create a set of rules and constraints that a given type of \ac{XML} document must follow in order to be considered valid. These rules are meant to create a contract on the type of information contained in the \ac{XML} documents, apart from having well formed \ac{XML}. To describe the rules and restrictions for a given \ac{XML} document the \ac{XSD} language relies on two types of data, elements and attributes. Elements are the most complex data type, they can contain other elements as children and can also have attributes. Attributes on the other hand are just pairs of information, defined by their name and their value. The value of a given attribute can be then restricted by multiple constraints existing on the language. There are multiple elements and attributes present in the \ac{XSD} language, which are specified at \href{http://www.datypic.com/sc/xsd/s-xmlschema.xsd.html}{XSD Schema}. In this thesis we will use the set of rules and restrictions of the \ac{XSD} files provided to build a fluent \ac{API} that will enforce the rules and restrictions specified by the given file.

\section{J2html} % (fold)
\label{sec:j2html}

What is it? What are the similarities? What are the improvements introduced by this dissertation that make it better or worse than j2html?

\section{Apache Velocity} % (fold)
\label{sec:apachevelocity}

What is it? What are the similarities? What are the improvements introduced by this dissertation that make it better or worse than apache velocity?
