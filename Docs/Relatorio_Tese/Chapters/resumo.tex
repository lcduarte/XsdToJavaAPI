\abstractPT  % Do NOT modify this line

%Introdução - Informar, em poucas palavras, o contexto em que o trabalho se insere, sintetizando a problemática estudada. 
Actualmente a utilização de linguagens de markup é recorrente no mundo da tecnologia, sendo o \ac{HTML} a linguagem mais utilizada graças à sua utilização no mundo da Web. Tendo isso em conta é necessário que existam ferramentas capazes de escrever documentos bem formados de forma eficaz. Actualmente essa tarefa é realizada por template engines, tendo como base ficheiros externos com templates de resposta, o que não garante que estes sejam bem formados e acrescenta o overhead do carregamento do ficheiro para memória.

% Objectivo - Deve ser explicitado claramente. 
\noindent
O nosso objectivo é criar as ferramentas necessárias para gerar \ac{API}s fluentes tendo em conta a sua definição sintática, expressa em \ac{XSD}, garantindo que as restrições dessa mesma linguagem são verificadas. A geração de \ac{API}s deve ser automatizada de modo a evitar erro humano e tornar a geração de código mais rápida. Automatizando a geração de \ac{API}s cria-se também uma abordagem uniforme às diferentes linguagens utilizadas.

%Métodos - Destacar os procedimentos metodológicos adoptados. 
\noindent
Para alcançar os nossos objectivos vai ser utilizada a linguagem Java para extrair informação sintática da linguagem do seu ficheiro de definição. Tendo essa informação em conta vão ser gerados bytecodes para refletir a definição da linguagem para a linguagem Java. Para implementar as restrições em Java é sempre prioritizada a validação de restrições em tempo de compilação, apenas validando em tempo de execução informação que não existe aquando da geração da \ac{API}.

%Resultados - Destacar os mais relevantes para os objectivos pretendidos. 
\noindent
Comparando a solução desenvolvida com soluções semelhantes, incluindo dez template engines e outra solução semelhante à que é apresentada, obtemos resultados favoráveis, verificando que a solução sugerida é a mais eficiente em todos os testes feitos. Estes resultados são importantes, especialmente considerando que apesar de ser a solução mais eficiente introduz também a verificação das restrições da linguagem utilizada tendo em conta a sua definição sintática.

% Keywords of abstract in Portuguese
\begin{keywords}
XML, XSD, Geração Automática de Código, API fluente \ldots
\end{keywords}
% to add an extra black line
