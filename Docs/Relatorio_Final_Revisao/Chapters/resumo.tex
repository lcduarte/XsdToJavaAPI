\abstractPT  % Do NOT modify this line

%Introdução - Informar, em poucas palavras, o contexto em que o trabalho se insere, sintetizando a problemática estudada. 
Actualmente a utilização de linguagens de \textit{markup} é recorrente no mundo da tecnologia, sendo o \ac{HTML} a linguagem mais utilizada graças à sua utilização no mundo da Web. Tendo isso em conta é necessário que existam ferramentas capazes de escrever documentos bem formados de forma eficaz. No entanto, a abordagem mais utilizada, \textit{template engines}, tem dois problemas principais: 1) não garante a geração de documentos bem formados, 2) não garante um bom desempenho, devido à utilização de ficheiros de texto como ficheiros de template.

% Objectivo - Deve ser explicitado claramente. 
\noindent
Para resolver o primeiro problema propomos que um template \ac{HTML} passe a ser definido como uma \textit{first-class function}. Para isto é necessário criar uma linguagem específica de domínio para que estas funções possam manipular a linguagem \ac{HTML}. O nosso objectivo principal é criar as ferramentas necessárias para gerar linguagens específicas de domínio com base no ficheiro de definição da linguagem, explícito num ficheiro \ac{XSD}. A linguagem de domínio gerada deve também garantir que as restrições da respectiva linguagem são verificadas. Removendo os ficheiros textuais que definem templates minimizam-se também os problemas de desempenho introduzidos pelo carregamento de ficheiros de texto e reduzem-se o número de operações sobre \texttt{Strings}.

%Métodos - Destacar os procedimentos metodológicos adoptados. 
\noindent
A minha proposta, chamada \texttt{xmlet}, inclui ferramentas que possibilitam: 1) a análise e a extração de informação de um ficheiro \ac{XSD}, 2) a geração de classes e métodos que definem uma linguagem específica de domínio que reflete as regras presentes no ficheiro \ac{XSD}, 3) a abstração da utilização da linguagem de domínio gerada, com a utilização do padrão Visitor. Para validar esta solução criaram-se linguagens de domínio não só para \ac{HTML} como também para a linguagem utilizada para definir layouts visuais para Android e para a linguagem das expressões regulares.

%Resultados - Destacar os mais relevantes para os objectivos pretendidos. 
\noindent
Comparando a solução desenvolvida com soluções semelhantes, incluindo \textit{template engines} e algumas soluções com inovações face à abordagem dos \textit{template engines}, obtemos resultados favoráveis. Verificamos que a solução sugerida é a mais eficiente em todos os testes feitos. Estes resultados são importantes, especialmente considerando que apesar de ser a solução mais eficiente introduz também a verificação das restrições da linguagem utilizada tendo em conta a sua definição sintática.

% Keywords of abstract in Portuguese
\begin{keywords}
XML, eXtensive Markup Language, XSD, eXtensive Markup Language Schema Definition, Geração Automática de Código, Interface Fluente, Linguage Specífica de Domínio.
\end{keywords}
% to add an extra black line
